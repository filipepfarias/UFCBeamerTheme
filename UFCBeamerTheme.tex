% !BIB program = biber
\documentclass[usenames,dvipsnames,aspectratio=149,8pt]{beamer}

\usepackage[utf8]{inputenc}

%% Presentation options
% comment/uncomment this line to enable/disable debug mode
%\newcommand*{\DEBUG}{}
% comment/uncomment this line to enable/disable notes pages
\newcommand*{\NotesPages}{}

\usepackage{beamerthemeufc} 
% Just for note {\scshape Smaller Title}
\title[\textbf{\uppercase{Smaller Title}}]{Example Presentation Created with the Beamer Package}
\subtitle{A subtitle example}
\author[F. P. Farias]{Filipe P. de Farias}
\institute[UFC]{Department of Teleinformatics Engineering\\Universidade Federal do Ceará}
\titlegraphic{\includegraphics[width=25pt]{graphics/brasao4_vertical_cor}}
\date{\today}

%% Bibliography
\usepackage[style=apa,backend=biber]{biblatex}
\addbibresource{bibliography.bibtex}

%%% Suppress biblatex annoying warning
\usepackage{silence}
\WarningFilter{biblatex}{Patching footnotes failed}

\begin{document}

\frame{\titlepage}

\section{Introduction}
\subsection{Overview of the Beamer Class}

\begin{frame}[fragile]
\justifying
\frametitle{Features of the Beamer Class}

This is a demonstration of {\bf\color{UFCOrange}UFC Beamer Theme} made by \insertshortauthor{}.

\begin{itemize}
\item<1-> Normal LaTeX class\parencite{article3}.
\item<2-> Easy overlays Easy overlays Easy overlays Easy overlays Easy overlays Easy overlays.
\item<3-> No external programs needed.      
\end{itemize}

\begin{columns}
\column{.5\textwidth}
This is a demonstration of two columns.
\column{.5\textwidth}
This is a demonstration of two columns.
\end{columns}
\[
  \begin{tikzpicture}[% from https://tex.stackexchange.com/a/330411/121799
    every left delimiter/.style={xshift=.75em},
    every right delimiter/.style={xshift=-.75em}
  ]
    \matrix[
      matrix of math nodes,
      left delimiter=(,
      right delimiter=),
      nodes in empty cells
    ] (m) {
      1 & ~~~ &  & 1 \\
      0 & & & \\
       & & & \\
      1 &  & 0& 1\\ 
    };
    \draw (m-1-1) -- (m-1-4);
    \draw (m-1-1) -- (m-4-4);
    \draw (m-2-1) -- (m-4-1);
    \draw (m-2-1.-20) -- (m-4-3);
    \draw (m-4-1) -- (m-4-3);
    \draw (m-1-4) -- (m-4-4);
  \end{tikzpicture}
\]
\end{frame}

\frame
{\frametitle{Features of the Beamer Class}

    \begin{block}{Remark}
    Sample text
    \end{block}
    
    \begin{alertblock}{Important theorem}
    Sample text in red box
    \end{alertblock}
    
    \begin{examples}
    Sample text in green box. The title of the block is ``Examples''.
    \end{examples}

}
\section{Second Section}
\subsection{Overview of the Beamer Class}
\frame
{
  \frametitle{Features of the Beamer Class}

    \begin{block}{Remark}
    Sample text
    \end{block}
    
    \begin{alertblock}{Important theorem}
    Sample text in red box
    \end{alertblock}
    
    \begin{examples}
    Sample text in green box. The title of the block is ``Examples''. \cite{article1}
    \end{examples}

}
\nocite{*}
\begin{frame}[allowframebreaks]{Bibliography}
\printbibliography
\end{frame}
\end{document}